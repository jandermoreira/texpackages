%
% DRAWDC: file
% This file is loaded by ../drawdc.sty
%


%%%%%%%%%%%%%%%%%%%%%%%%%%%%%%%%%%%%%%%%%%%%%%%%%%%%%%%%%%%%%%%%
% Variables
%

\newlength{\ds@file@recordwidth}
\newlength{\ds@file@recordheight}
\newlength{\ds@file@nodedistance}
\newlength{\ds@file@recordlinewidth}
\newlength{\ds@file@blocklinewidth}
\newlength{\ds@file@linktolinkdistance}
\def\ds@file@linecolor{.}
\def\ds@file@fieldwidthlist{2cm, 5cm, 3cm, 10cm, 2cm}
\newif\ifds@file@showrrn
\def\ds@file@numberoffields{5}

\newcommand{\ds@file@defaults}{
	\def\ds@file@filename{file}%
	\setlength{\ds@file@recordwidth}{2cm}
	\setlength{\ds@file@recordheight}{0.5cm}
	\setlength{\ds@file@nodedistance}{0.05cm}
	\def\ds@file@recordsperblock{8}
	\def\ds@file@blockfont{\normalsize\rmfamily\itshape}
	\def\ds@file@recordfont{\normalsize\ttfamily}
	\setlength{\ds@file@recordlinewidth}{0.3pt}
	\setlength{\ds@file@blocklinewidth}{1pt}
	\setlength{\ds@file@linktolinkdistance}{4pt}
	\ds@file@showrrnfalse
}
\ds@file@defaults

%%%%%%%%%%%%%%%%%%%%%%%%%%%%%%%%%%%%%%%%%%%%%%%%%%%%%%%%%%%%%%%%
% Parameters
%
\pgfkeys{
	/dsdraw/file/.cd,
	% ** list properties
	start at/.store in = \ds@file@startat,
	name/.store in = \ds@file@filename,
	%
	field widths/.store in = \ds@file@fieldwidthlist,
	%
	record width/.code = {\setlength{\ds@file@recordwidth}{#1}},
	record height/.code = {\setlength{\ds@file@recordheight}{#1}},
	record font/.store in = \ds@file@recordfont,
	records per block/.store in = \ds@file@recordsperblock,
	%
	start at/.store in = \ds@file@currentxy,	
	%
	options/.code = {\def\ds@file@options{/tikz/.cd, #1}},
	%
	show rrn/.code = {\ds@file@showrrntrue},
	hide rrn/.code = {\ds@file@showrrnfalse},
}

%%%%%%%%%%%%%%%%%%%%%%%%%%%%%%%%%%%%%%%%%%%%%%%%%%%%%%%%%%%%%%%%
% Drawing stuff
%
\usetikzlibrary{calc, chains, scopes, math, fit, shapes.multipart}
\tikzset{
	every record/.style = {
		draw = \ds@file@linecolor,
		inner sep = 2pt,
		outer sep = 0pt,
		text width = \dimexpr \ds@file@recordwidth \relax,
		minimum height = \ds@file@recordheight,
%		text depth = 0.005\ds@file@recordheight,
		font = \ds@file@recordfont,
		align = left,
		line width = \ds@file@recordlinewidth,
		rectangle split,
		rectangle split horizontal,
		rectangle split parts = \ds@file@numberoffields,
		rectangle split allocate boxes = \ds@file@numberoffields,
		rectangle split empty part width = \dimexpr \ds@file@recordwidth - 4.3pt \relax,
	},
	rnn index/.style = {
		draw = none,
		font = \footnotesize\itshape,
		align = right,
		anchor = east,
	},
	every fs link/.style = {
		every link,
		rounded corners,
	},
}


\newcommand{\dsfilelink}[3][1]{
	\draw[every fs link] #2 -|
		($#3 + (\dimexpr 1.8cm + #1\ds@file@linktolinkdistance, 0)$) --
		#3;
}



\newcommand{\ds@file@settanchor}[1]{
	\anchor{t#1}{
%		\southwest
		\northeast
		\pgf@ya=\pgf@y
		\pgf@y=\dimexpr \pgf@ya - 14\ds@file@nodedistance 
			- \ds@file@linktolinkdistance*#1 \relax
	}
}
\newcommand{\ds@file@setbanchor}[1]{
	\anchor{b#1}{
		\southwest
		\pgf@ya=\pgf@y
		\northeast
		\pgf@y=\dimexpr \pgf@ya + 4\ds@file@nodedistance 
			+ \ds@file@linktolinkdistance*#1 \relax
	}
}
\pgfdeclareshape{fsblock}{
	\inheritsavedanchors[from=rectangle]
	\inheritanchorborder[from=rectangle]
	\foreach \anchor in {north, north west, north east, center, eest, east,
		mid, midwest, mid east, base, base west, base east, south, west,
		south east, south west}{\inheritanchor[from=rectangle]{\anchor}}
	\inheritbackgroundpath[from=rectangle]
	\foreach \l in {0, ..., 5}{
		\expandafter\ds@file@settanchor\expandafter{\l}
		\expandafter\ds@file@setbanchor\expandafter{\l}
	}
}


%%%%%%%%%%%%%%%%%%%%%%%%%%%%%%%%%%%%%%%%%%%%%%%%%%%%
% processing lists

\NewDocumentCommand\ds@processrecordlist{%
	>{\SplitList{,}}m%
}{%
	\ProcessList{#1}{\ds@processrecord}%
}

\NewDocumentCommand\ds@processblocklist{%
	>{\SplitList{;}}m%
}{%
	\ProcessList{#1}{\ds@processblock}%
}



\newcounter{fieldcounter}
\newcounter{blockcounter}
\newcounter{recordcounter}


%%%%%%%%%%%%%%%%%%%%%%%%%%%%%%%%%%%%%%%%
% fields

%********************** https://tex.stackexchange.com/questions/247821/define-a-new-rectangular-node-with-several-anchor-points-in-tikz

\newcommand{\ds@numbername}[1]{%
	\ifcase#1\relax
		\or %1
		\or two%
		\or three%
		\or four%
		\or five%
		\or six%
		\or seven%
		\or eight%
		\or nine%
		\or ten%
		\or eleven%
		\or twelve%
		\or thirteen%
		\or fourteen%
		\or fifteen%
		\or sixteen%
		\else ErroNoNumero%
	\fi
}
\def\ds@comma{}
\newcommand{\ds@processfield}[1]{%
	\ds@comma#1%
	\global\def\ds@comma{, }%
}

% begin https://tex.stackexchange.com/a/297521/64821
\ExplSyntaxOn
\NewDocumentCommand{\ds@addtolist}{mm}{
	% #1 = items; #2 = list name
	\seq_clear_new:c { dd_list_#2_seq }
	\clist_map_inline:nn { #1 }{
		\seq_put_right:cn { dd_list_#2_seq } { ##1 }
	}
}
\NewDocumentCommand{\ds@addonetolist}{mm}{
	% #1 = items; #2 = list name
	\seq_put_right:cn { dd_list_#2_seq } { #1 }
}
\NewDocumentCommand{\ds@twolistloop}{mmm}{
	% #1 = first list name; #2 = second list name; #3 = two argument macro
	\seq_mapthread_function:ccN
		{ dd_list_#1_seq } { dd_list_#2_seq } #3
}
\ExplSyntaxOff
% end https://tex.stackexchange.com/a/297521/64821


\newcommand{\ds@makefield}[2]{%
	#1%
	\stepcounter{fieldcounter}%
	\nodepart[text width = #2]{\ds@numbername{\thefieldcounter}}%
}


\newcommand{\ds@fieldprocessing}[2]{%
	\setcounter{fieldcounter}{0}%
	\def\fulllist{#1}%
	\toks0=\expandafter{#1}%
	\expandafter\ds@addtolist\expandafter{\the\toks0}{fields}%
	\ifnum\thefieldcounter<\ds@file@numberoffields%
		\foreach \i in {\thefieldcounter, ..., \ds@file@numberoffields}{
			\ds@addonetolist{x}{fields}%
		}%
	\fi%
	\toks2=\expandafter{#2}%
	\expandafter\ds@addtolist\expandafter{\the\toks2}{widths}%
	\ds@twolistloop{fields}{widths}{\ds@makefield}%
	\setcounter{fieldcounter}{1}%
	\stepcounter{fieldcounter}%
}


%%%%%%%%%%%%%%%%%%%%%%%%%%%%%%%%%%%%%%%%
% records
\newcommand{\ds@file@drawrecord}[2][]{
	\node[on chain] (x) {x};
		\begin{dsrecord*}
			\dsfieldfixed*{jander}{10}
		\end{dsrecord*}
%	\def\extraopt{#1}
%	\ifds@file@showrrn
%		\let\@saveshowrnn{\global\ds@file@showrrntrue}
%	\else
%		\let\@saveshowrnn{\global\ds@file@showrrnfalse}
%	\fi
%	\IfStrEq{#2}{skip}{
%		\def\@reclabel{\tiny\vdots}
%		\def\extraopt{#1, draw = none, text height = 0.5\ds@file@recordheight, align = center}
%	}{
%		\IfStrEq{#2}{empty}{
%			\def\@reclabel{}
%			\def\extraopt{#1, draw = none, text height = 0.5\ds@file@recordheight, align = center}
%			\global\ds@file@showrrnfalse
%		}{
%			\IfStrEq{#2}{emptysymb}{
%				\def\@reclabel{\NullSymb}
%			}{
%				\def\@reclabel{#2%
%					\toks0=\expandafter{#2}%
%					\toks2=\expandafter{\ds@file@fieldwidthlist}%
%					\ds@fieldprocessing{\the\toks0}{\the\toks2}%
%				}
%			}
%		}
%	}
%	{%group
%		\dstikzset{every record/.append style = {\extraopt}}
%		\ifds@file@isfirst
%			\node[yshift = \@yshift,
%				on chain, every record, #1]
%				(\ds@file@filename-\theblockcounter-\therecordcounter)
%				{\@reclabel};
%			\global\ds@file@isfirstfalse
%		\else
%			\node [on chain, every record, #1] 	(\ds@file@filename-\theblockcounter-\therecordcounter)
%			{\@reclabel};
%		\fi
%		\ifds@file@showrrn
%			\pgfmathparse{int(\theblockcounter *
%				\ds@file@recordsperblock + \therecordcounter)}
%			\node[rnn index] at ($(\ds@file@filename-\theblockcounter-\therecordcounter.west) - (0.25cm , 0)$) {\pgfmathresult};
%		\fi
%	}
%	\stepcounter{recordcounter}
%	\@saveshowrnn
}

\newcommand{\ds@processrecord}[1]{
	\IfStrEq{#1}{stop}
		{\global\@stoprecordstrue}
		{
			\toks0=\expandafter{#1}
			\expandafter\ds@file@drawrecord\expandafter{\the\toks0}
		}
}


%%%%%%%%%%%%%%%%%%%%%%%%%%%%%%%%%%%%%%%%%%%%%%%%%%%%%%
% blocks
\newif\if@stoprecords
\newcommand{\ds@processblock}[1]{
%	\IfStrEq{#1}{\@label}{\def\@label{}}{}
	\IfBeginWith{#1}{skip}{
		\StrCut{#1}{ to }\@skip\@blocknumber
		\ifx\@blocknumber\empty
			\stepcounter{blockcounter}
		\else
			\setcounter{blockcounter}{\@blocknumber}
		\fi
		\node[on chain] {\scriptsize\vdots};
	}{
		\global\ds@file@isfirsttrue
		\setcounter{recordcounter}{0}
		\@stoprecordsfalse
		\toks0=\expandafter{#1}
		\expandafter\ds@processrecordlist\expandafter{\the\toks0}
		\pgfmathsetmacro\@remainingrecords{int(\ds@file@recordsperblock -
			\therecordcounter)}
		\pgfmathsetmacro\@lastrecord{int(\ds@file@recordsperblock - 1)}
		\ifnum\@remainingrecords>0
			\if@stoprecords
				\foreach \rec in {1, ..., \@remainingrecords}{
					\ds@file@drawrecord[draw = none]{empty}
				}
			\else
				\foreach \rec in {1, ..., \@remainingrecords}{
					\ds@file@drawrecord{}
				}
			\fi
		\fi
		
		% block coordinates
		\coordinate (nw) at ($(\ds@file@filename-\theblockcounter-0.north west) + (-3\ds@file@nodedistance, 2\ds@file@nodedistance)$);
		\coordinate (se) at ($(\ds@file@filename-\theblockcounter-\@lastrecord.south east) +
				(3\ds@file@nodedistance, -2\ds@file@nodedistance)$);
		\node[fsblock, draw, line width = \ds@file@blocklinewidth,
			fit = {(nw) (se)}, inner sep = 0pt]
			(\ds@file@filename-\theblockcounter) {};
		\IfStrEq{}{}{\def\@text{bloco $\theblockcounter$}}{\def\@text{XXXXX}}
		\node [font = \ds@file@blockfont, anchor = west] at
			($(\ds@file@filename-\theblockcounter-0.north east) + (2.\ds@file@nodedistance, -4\ds@file@nodedistance)$) {\@text\strut};
		\stepcounter{blockcounter}
		\global\def\@yshift{-3\ds@file@nodedistance}
	}
}


%%%%%%%%%%%%%%%%%%%%%%%%%%%%%%%%%%%%%%%%%%%%%%%%%%%%%5
% file
\newif\ifds@file@isfirst
\newcommand{\dsfile}[2][]{{
	\pgfkeys{
		/drawdc/file/.cd,
		#1,
	}
	{
		\dstikzset{%
			start chain = \ds@file@filename{} going below,
			node distance = \ds@file@nodedistance,
		}
		\setcounter{blockcounter}{0}
		\def\@yshift{0pt}
		\toks0=\expandafter{#2}
		\ds@processblocklist{#2}
	}
}}


%%%%%%%%%%%%%%%%%%%%%%%%%%%%%%%%%%%%%%%%%%%%%%%%%%%%%%%%%%%%%%%%
% Utilities
%

% dsfileset: shortcut to file-related pgfkeys
\newcommand{\dsfileset}[1]{
	\pgfkeys{/drawdc/file/.cd, #1}
}

%% EOF