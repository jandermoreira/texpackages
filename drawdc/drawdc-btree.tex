%
% DRAWDC: btree
% This file is loaded by ../drawdc.sty
%

\RequirePackage{xcolor}

%%%%%%%%%%%%%%%%%%%%%%%%%%%%%%%%%%%%%%%%%%%%%%%%%%%%%%%%%%%%%%%%
% Variables
%



%%%%%%%%%%%%%%%%%%%%%%%%%%%%%%%%%%%%%%%%%%%%%%%%%%%%%%%%%%%%%%%%
% Styles and code
%

\RequirePackage{tikz}
\usetikzlibrary{arrows, shapes, trees, calc, positioning}

\tikzset{
	key/.style = {
		circle,
		fill = white,
		dotted,
		thick,
		draw,
	},
	vazio/.style = {
		draw = none,
		fill = none,
	},
	no/.style = {
		rectangle split,
		rectangle split horizontal,
		rectangle split parts = #1,
		rectangle split draw splits = false,
		rounded corners,
		fill = black!15,
		draw = black,
		thick,
		anchor = center,
		minimum height = 1.8em,
	},
	no b/.style = {
		rectangle split,
		rectangle split horizontal,
		rectangle split parts = #1,
		rectangle split empty part width = 1.3ex,
		rectangle split draw splits = false,
		rounded corners,
		fill = black!15,
		draw = black,
		thick,
		anchor = center,
		minimum height = 1.8em,
	}, 
	edge from parent/.style = {
		thick,
		-latex,
		draw = black!95,
	},
	child/.style = {
		edge from parent path = {(\tikzparentnode.west)
			++(0.06cm+#1*0.65cm, -0.18cm) -- (\tikzchildnode)},
	},
	nil/.style = {
		edge from parent = {thick, -square, draw = black},
	},
}

%%%%%%%%%%%%%%%%%%%%%%%%%%%%%%%%%%%%%%%%%%%%%%%%%%%%%%%%%%%%
% Variables

\newlength{\drawdc@treewidth}
\newcounter{drawdc@treelevels}
%%%%%%%%%%%%%%%%%%%%%%%%%%%%%%%%%%%%%%%%%%%%%%%%%%%%%%%%%%%%
% Btree drawing macros

% draw a btree
\newcommand{\drawdc@btdraw}[1]{%
	% #1: tree name
	\par --- end of #1\par
}


% create a btree
\newenvironment{btree}[1]{%
	% #1: tree name
	--- start of #1\par
	\def\drawdc@treename{#1}
	\setlength{\drawdc@treewidth}{0pt}%
	\setcounter{drawdc@treelevels}{0}%
}{%
	\drawdc@btdraw{\drawdc@treename}%
}

% add root
\newcommand{\ddbtroot}{%
	\setcounter{drawdc@treeleves}{1}%
	
}

%% EOF