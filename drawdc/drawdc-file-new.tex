%
% DRAWDC: file
% This file is loaded by ../drawdc.sty
%

\RequirePackage{tikz}
\RequirePackage{xstring}
\RequirePackage{xparse}

%%%%%%%%%%%%%%%%%%%%%%%%%%%%%%%%%%%%%%%%%%%%%%%%%%%%%%%%%%%%%%%%
% Parameters
%
\pgfkeys{
	/drawdc/file struct/.cd,
	% ** list properties
	start at/.store in = \dd@filestruct@startat,
	name/.store in = \dd@filestruct@filename,
	%
	field widths/.code = {\dd@filestruct@setwidths{#1}},
	fields per record/.store in = \dd@filestruct@numberoffields,
%			.code = {\dd@filestruct@setfieldsperrecord{#1}},
	%
	record width/.code = {\setlength{\dd@filestruct@fieldwidth}{\dimexpr
		#1/\dd@filestruct@numberoffields - 4.3pt\relax}},
	record height/.code = {\setlength{\dd@filestruct@recordheight}{#1}},
	record font/.store in = \dd@filestruct@recordfont,
	records per block/.store in = \dd@filestruct@recordsperblock,
	%
%	start at/.store in = \dd@filestruct@currentxy,	
	%
	options/.code = {\def\dd@filestruct@options{/tikz/.cd, #1}},
	%
	show rrn/.code = {\dd@filestruct@showrrntrue},
	hide rrn/.code = {\dd@filestruct@showrrnfalse},
	%
	hide borders/.code = {\dd@filestruct@showbordersfalse},
	show borders/.code = {\dd@filestruct@showborderstrue},
	%
	hide block labels/.code = {\dd@filestruct@showblocklabelsfalse},
	show block labels/.code = {\dd@filestruct@showblocklabelstrue},	
}


%%%%%%%%%%%%%%%%%%%%%%%%%%%%%%%%%%%%%%%%%%%%%%%%%%%%%%%%%%%%%%%%
% Variables
%

\newlength{\dd@filestruct@fieldwidth}
\newlength{\dd@filestruct@recordheight}
\newlength{\dd@filestruct@nodedistance}
\newlength{\dd@filestruct@recordlinewidth}
\newlength{\dd@filestruct@blocklinewidth}
\newlength{\dd@filestruct@linktolinkdistance}
\def\dd@filestruct@linecolor{.}
\def\dd@filestruct@fieldwidthlist{\dd@filestruct@fieldwidth}
\newif\ifdd@filestruct@showrrn
\def\dd@filestruct@numberoffields{1}
\newif\ifdd@filestruct@showborders
\newif\ifdd@filestruct@showblocklabels


%% defaults
\pgfkeys{/drawdc/file struct/.cd,
	record width = 5cm,
	fields per record = 1,
	records per block = 8,
	record height = 0.5cm,
	record font = \normalsize\ttfamily,
	show rrn,
}
\setlength{\dd@filestruct@nodedistance}{0.05cm}
\def\dd@filestruct@blockfont{\normalsize\rmfamily\itshape}
\setlength{\dd@filestruct@recordlinewidth}{0.3pt}
\setlength{\dd@filestruct@blocklinewidth}{1pt}
\setlength{\dd@filestruct@linktolinkdistance}{4pt}
\dd@filestruct@showborderstrue
\dd@filestruct@showblocklabelstrue

%%%%%%%%%%%%%%%%%%%%%%%%%%%%%%%%%%%%%%%%%%%%%%%%%%%%%%%%%%%%%%%%
% Drawing stuff
%
\usetikzlibrary{calc, chains, scopes, math, fit, shapes.multipart}
\tikzset{
	every record/.style = {
		draw = \dd@filestruct@linecolor,
		inner sep = 2pt,
		outer sep = 0pt,
		text width = \dimexpr \dd@filestruct@fieldwidth \relax,
		minimum height = \dd@filestruct@recordheight,
%		text depth = 0.005\dd@filestruct@recordheight,
		font = \dd@filestruct@recordfont,
		align = left,
		line width = \dd@filestruct@recordlinewidth,
		rectangle split,
		rectangle split horizontal,
		rectangle split parts = \dd@filestruct@numberoffields,
		rectangle split allocate boxes = \dd@filestruct@numberoffields,
		rectangle split empty part width = \dimexpr \dd@filestruct@fieldwidth - 4.3pt \relax,
	},
	rnn index/.style = {
		draw = none,
		font = \footnotesize\itshape,
		align = right,
		anchor = east,
	},
	every fs link/.style = {
		every link,
		rounded corners,
	},
}


\newcommand{\ddfslink}[3][1]{
	\draw[every fs link] #2 -|
		($#3 + (\dimexpr 1.8cm + #1\dd@filestruct@linktolinkdistance, 0)$) --
		#3;
}


\newlength{\@fwidth}
\newcommand{\dd@filestruct@setwidths}[1]{%
%	\noexpandarg%
%	\def\@firstwidth{\StrBefore{#1}{,}}%
%	\edef\@len{\StrLen{\@firswidth}}%
%	\ifnum\@len=0
%		\node at (2,2) {#1: empty (\firstwidth) \@len};%
%		\setlength{\dd@filestruct@fieldwidth}{#1}%
%%	\else
%%		\node at (2,2) {valor: \@firstwidth};%
%%		\setlength{\dd@filestruct@fieldwidth}{2cm}%
%	\fi
	\def\dd@filestruct@fieldwidthlist{#1}%
}

\newlength{\@recwidth}
\newcommand{\dd@filestruct@setfieldsperrecord}[1]{%
%	\setlength{\@recwidth}{\dimexpr \dd@filestruct@numberoffields *
%		\dd@filestruct@fieldwidth\relax}%
%	\dd@filestruct@setwidths{\@recwidth}%
	\def\dd@filestruct@numberoffields{#1}%
}

\newcommand{\dd@filestruct@settanchor}[1]{
	\anchor{t#1}{
%		\southwest
		\northeast
		\pgf@ya=\pgf@y
		\pgf@y=\dimexpr \pgf@ya - 14\dd@filestruct@nodedistance 
			- \dd@filestruct@linktolinkdistance*#1 \relax
	}
}
\newcommand{\dd@filestruct@setbanchor}[1]{
	\anchor{b#1}{
		\southwest
		\pgf@ya=\pgf@y
		\northeast
		\pgf@y=\dimexpr \pgf@ya + 4\dd@filestruct@nodedistance 
			+ \dd@filestruct@linktolinkdistance*#1 \relax
	}
}
\pgfdeclareshape{fsblock}{
	\inheritsavedanchors[from=rectangle]
	\inheritanchorborder[from=rectangle]
	\foreach \anchor in {north, north west, north east, center, eest, east,
		mid, midwest, mid east, base, base west, base east, south, west,
		south east, south west}{\inheritanchor[from=rectangle]{\anchor}}
	\inheritbackgroundpath[from=rectangle]
	\foreach \l in {0, ..., 5}{
		\expandafter\dd@filestruct@settanchor\expandafter{\l}
		\expandafter\dd@filestruct@setbanchor\expandafter{\l}
	}
}


%%%%%%%%%%%%%%%%%%%%%%%%%%%%%%%%%%%%%%%%%%%%%%%%%%%%
% processing lists
\NewDocumentCommand\dd@fs@processrecordlist{%
	>{\SplitList{!}}m%
}{%
	\ProcessList{#1}{\dd@fs@processrecord}%
}

\NewDocumentCommand\dd@fs@processblocklist{%
	>{\SplitList{;}}m%
}{%
	\ProcessList{#1}{\dd@fs@processblock}%
}



\newcounter{fieldcounter}
\newcounter{blockcounter}
\newcounter{recordcounter}


%%%%%%%%%%%%%%%%%%%%%%%%%%%%%%%%%%%%%%%%
% fields

%********************** https://tex.stackexchange.com/questions/247821/define-a-new-rectangular-node-with-several-anchor-points-in-tikz

\newcommand{\dd@numbername}[1]{%
	\ifcase#1\relax
		\or %1
		\or two%
		\or three%
		\or four%
		\or five%
		\or six%
		\or seven%
		\or eight%
		\or nine%
		\or ten%
		\or eleven%
		\or twelve%
		\or thirteen%
		\or fourteen%
		\or fifteen%
		\or sixteen%
		\else ErroNoNumero%
	\fi%
}
\def\dd@comma{}
\newcommand{\dd@processfield}[1]{%
	\dd@comma#1%
	\global\def\dd@comma{, }%
}

% begin https://tex.stackexchange.com/a/297521/64821
\ExplSyntaxOn
\NewDocumentCommand{\dd@addtolist}{mm}{
	% #1 = items; #2 = list name
	\seq_clear_new:c { dd_list_#2_seq }
	\clist_map_inline:nn { #1 }{
		\seq_put_right:cn { dd_list_#2_seq } { ##1 }
	}
}
\NewDocumentCommand{\dd@addonetolist}{mm}{
	% #1 = items; #2 = list name
	\seq_put_right:cn { dd_list_#2_seq } { #1 }
}
\NewDocumentCommand{\dd@twolistloop}{mmm}{
	% #1 = first list name; #2 = second list name; #3 = two argument macro
	\seq_mapthread_function:ccN
		{ dd_list_#1_seq } { dd_list_#2_seq } #3
}

\ExplSyntaxOff
% end https://tex.stackexchange.com/a/297521/64821

\newcommand{\dd@makefield}[2]{%
	\ifnum\thefieldcounter>1%
		\nodepart[text width = #2]{\dd@numbername{\thefieldcounter}}%
	\fi%
	#1%(#2)%
	\stepcounter{fieldcounter}%
}


\newcommand{\dd@fs@fieldprocessing}[1]{%
%	\toks0=\expandafter{#1}%
%	\expandafter\dd@addtolist\expandafter{\the\toks0}{fields}%
	\dd@addtolist{#1}{fields}%
	\setcounter{fieldcounter}{\dd@countlist{#1}}%
	\@whilenum\thefieldcounter<\dd@filestruct@numberoffields\do{%
		\dd@addonetolist{}{fields}%
		\stepcounter{fieldcounter}%
	}%
	\toks0=\expandafter{\dd@filestruct@fieldwidthlist}%
	\expandafter\dd@addtolist\expandafter{\the\toks0}{widths}%
	\setcounter{fieldcounter}{1}%
	\global\def\dd@partnumber{}
	\dd@twolistloop{fields}{widths}{\dd@makefield}%
}


%%%%%%%%%%%%%%%%%%%%%%%%%%%%%%%%%%%%%%%%
% records
\newcommand{\dd@filestruct@drawrecord}[2][]{
	\def\extraopt{#1}
	\ifdd@filestruct@showrrn
		\def\@saveshowrnn{\global\dd@filestruct@showrrntrue}
	\else
		\def\@saveshowrnn{\global\dd@filestruct@showrrnfalse}
	\fi
	\IfStrEq{#2}{skip}{
		\def\@reclabel{\tiny\vdots}
		\def\extraopt{#1, draw = none,
			text height = 0.5\dd@filestruct@recordheight, align = center}
	}{
		\IfStrEq{#2}{empty}{
			\def\@reclabel{}
			\def\extraopt{#1, draw = none, text height = 0.5\dd@filestruct@recordheight, align = center}
			\global\dd@filestruct@showrrnfalse
		}{
			\IfStrEq{#2}{emptysymb}{
				\def\@reclabel{\EmptySymb}
			}{
				\def\@reclabel{
					\toks0=\expandafter{#2}%
					\toks2=\expandafter{\dd@filestruct@fieldwidthlist}%
					\dd@fs@fieldprocessing{#2}%
				}
			}
		}
	}
	{%group
		\ddtikzset{every record/.append style = {\extraopt}}
		\ifdd@filestruct@isfirst
			\node[yshift = \@yshift,
				on chain, every record, #1]
				(\dd@filestruct@filename-\theblockcounter-\therecordcounter)
				{\@reclabel};
			\global\dd@filestruct@isfirstfalse
		\else
			\node [on chain, every record, #1]	(\dd@filestruct@filename-\theblockcounter-\therecordcounter)
			{\@reclabel};
		\fi
		\ifdd@filestruct@showrrn
			\pgfmathparse{int(\theblockcounter *
				\dd@filestruct@recordsperblock + \therecordcounter)}
			\node[rnn index] at ($(\dd@filestruct@filename-\theblockcounter-\therecordcounter.west) - (0.25cm , 0)$) {\pgfmathresult};
		\fi
	}
	\stepcounter{recordcounter}
	\@saveshowrnn
}

\newcommand{\dd@fs@processrecord}[1]{
	\IfStrEq{#1}{stop}
		{\global\@stoprecordstrue}
		{
%			\toks0=\expandafter{#1}
			\dd@filestruct@drawrecord{#1}
		}
}


%%%%%%%%%%%%%%%%%%%%%%%%%%%%%%%%%%%%%%%%%%%%%%%%%%%%%%
% blocks
\newif\if@stoprecords
\newcommand{\dd@fs@processblock}[1]{
	\begingroup
	\noexpandarg
%	\IfStrEq{#1}{\@label}{\def\@label{}}{}
	\IfBeginWith{#1}{skip}{
		\StrCut{#1}{ to }\@skip\@blocknumber
		\ifx\@blocknumber\empty
			\stepcounter{blockcounter}
		\else
			\setcounter{blockcounter}{\@blocknumber}
		\fi
		\node[on chain] {\scriptsize\vdots};
	}{
		\global\dd@filestruct@isfirsttrue
		\setcounter{recordcounter}{0}
		\@stoprecordsfalse
%		\toks0=\expandafter{#1}
		\dd@fs@processrecordlist{#1}
		\pgfmathsetmacro\@remainingrecords{int(\dd@filestruct@recordsperblock -
			\therecordcounter)}
		\pgfmathsetmacro\@lastrecord{int(\dd@filestruct@recordsperblock - 1)}
		\ifnum\@remainingrecords>0
			\if@stoprecords
				\foreach \rec in {1, ..., \@remainingrecords}{
					\dd@filestruct@drawrecord[draw = none]{empty}
				}
			\else
				\foreach \rec in {1, ..., \@remainingrecords}{
					\dd@filestruct@drawrecord{}
				}
			\fi
		\fi
		
		% block coordinates
		\coordinate (nw) at ($(\dd@filestruct@filename-\theblockcounter-0.north west) + (-3\dd@filestruct@nodedistance, 2\dd@filestruct@nodedistance)$);
		\coordinate (se) at ($(\dd@filestruct@filename-\theblockcounter-\@lastrecord.south east) +
				(3\dd@filestruct@nodedistance, -2\dd@filestruct@nodedistance)$);
		\node[fsblock, draw, line width = \dd@filestruct@blocklinewidth,
			fit = {(nw) (se)}, inner sep = 0pt]
			(\dd@filestruct@filename-\theblockcounter) {};
		\IfStrEq{}{}{\def\@text{bloco $\theblockcounter$}}{\def\@text{XXXXX}}
		\ifdd@filestruct@showblocklabel
			\node [font = \dd@filestruct@blockfont, anchor = west] at
				($(\dd@filestruct@filename-\theblockcounter-0.north east) + (2.\dd@filestruct@nodedistance, -4\dd@filestruct@nodedistance)$) {\@text\strut};
		\fi
		\stepcounter{blockcounter}
		\global\def\@yshift{-3\dd@filestruct@nodedistance}
	}
	\endgroup
}


%%%%%%%%%%%%%%%%%%%%%%%%%%%%%%%%%%%%%%%%%%%%%%%%%%%%%5
% file
\newif\ifdd@filestruct@isfirst
\newcommand{\ddfsfile}[2][]{
	\def\dd@filestruct@filename{file}%
	\pgfkeys{
		/drawdc/file struct/.cd,
		#1,
	}
	{
		\ddtikzset{%
			start chain = \dd@filestruct@filename{} going below,
			node distance = \dd@filestruct@nodedistance,
		}
		\setcounter{blockcounter}{0}
		\def\@yshift{0pt}
		\dd@fs@processblocklist{#2}
	}
}





\newcommand{\ddfsset}[1]{
	\pgfkeys{/drawdc/file struct/.cd, #1}
}

%% EOF