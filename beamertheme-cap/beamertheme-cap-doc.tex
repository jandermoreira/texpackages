%%%%%%%%%%%%%%%%%%%%%%%%%%%%%%%%%%%%%%%%%%%%%%%%%%%%%%%%%%%%%%%%%%%%%%%%
%
%  Documento exemplo para uso do tema do CAP no Beamer
%  Jander, 2018
%
\documentclass[
	brazilian, % idioma
	12pt, % pode ser 10pt, 11pt, 12pt...
	%aspectratio = 169, % 16x9 = widescreen
]{beamer}
\usepackage[utf8]{inputenc}
\usepackage{babel}
\usepackage[T1]{fontenc}
\usepackage{amsmath}
\usepackage{amsfonts}
\usepackage{amssymb}

% Tema: CAP
\usetheme{cap}


%%%%%%%%%%%%%%%%%%%%%%%%%%%%%%%%%%%%%%%%%%%%%%%%%%%%%%%%%%%%%%%%%%%%%%%%
% Agenda no inicio de cada seção (remova comentário de uma das opções)
%
%\agendaautomatica	% agenda em uma coluna
%\agendaautomatica[4]	% agenda em duas colunas, quebrando na seção indicada


%%%%%%%%%%%%%%%%%%%%%%%%%%%%%%%%%%%%%%%%%%%%%%%%%%%%%%%%%%%%%%%%%%%%%%%%
% Opção para geração de eslaides com notas
% O programa pympress consegue mostrar os eslaides no projetor e as notas na tela
%
%\usepackage{pgfpages}
%\setbeameroption{show notes on second screen=right}



%%%%%%%%%%%%%%%%%%%%%%%%%%%%%%%%%%%%%%%%%%%%%%%%%%%%%%%%%%%%%%%%%%%%%%%%
% Documentação
\title{Título da apresentação}
\author{Fulano de Tal}
\email{fulano@de.tal} %  LALIC; use \email{} para omitir
\dataapresentacao{25}{12}{2027} %  LALIC; argumentos: dia, mês e ano (numéricos)
\date{\today} % \date{} sumprime a data


%%%%%%%%%%%%%%%%%%%%%%%%%%%%%%%%%%%%%%%%%%%%%%%%%%%%%%%%%%%%%%%%%%%%%%%%
% Conteúdo
%
\begin{document}
\begin{frame}
	\titlepage	
\end{frame}

% Opcionalmente, comente este frame caso use \agendaautomatica
\begin{frame}{Agenda}
	\tableofcontents
\end{frame}


%%%%%%%%%%%%%%%%%%%%%%%%%%%%%%%%%%%%%%%%%%%%%%%%%%%%%%%%%%%%%%%%%%%%%%%%
% Seções, subseções...
% > texto normal: nome usado na \tableofcontents
% > texto curto (opcional): nome usado na navegação
%
% \section[texto curto]{texto normal} cria uma seção
% \subsection[texto curto]{texto normal} cria uma subseção

\section{Introdução e motivação}
\begin{frame}
	Eslaide sem título
\end{frame}

\begin{frame}{Introdução}
	Eslaide com título
\end{frame}

\begin{frame}{Introdução}{Motivação}
	Eslaide com título e subtítulo
\end{frame}


\section[Recursos]{Recursos de apresentação}

\subsection{Listas}

\begin{frame}{Listas}{Listas sem numeração}
	Lista sem numeração
	
	\begin{itemize}
		\item Um one eins
		\item Dois two zwei
		\item Três three drei
			\begin{itemize}
				\item Três ponto um
				\item Três ponto dois
			\end{itemize}
		\item Quatro four vier
	\end{itemize}
\end{frame}

\begin{frame}{Listas}{Listas com numeração}
	Lista com numeração
	
	\begin{enumerate}
		\item Um one eins
		\item Dois two zwei
		\item Três three drei
			\begin{itemize}
				\item Três ponto um
				\item Três ponto dois
			\end{itemize}
		\item Quatro four vier
	\end{enumerate}
\end{frame}

\subsection{Duas colunas}

\begin{frame}
	Texto em duas colunas (tamanhos diferentes\ldots)
	
	\begin{columns}
		\begin{column}{0.4\linewidth}
			Dewar's scotch whisky gin sour lime rickey balvenie satan's whiskers cheeky vimto ben nevis, sangría. Bunnahabhain the goldeneye redline vodka sundowner aviation, hurricane speyburn, glen ord.
		\end{column}
		\begin{column}{0.6\linewidth}
			 Sake screwdriver; mojito bloody aztec old mr. boston belvedere knockeen hills irish poteen. Myers gin sour anisette moonwalk glen garioch vodka sunrise glen spey cragganmore fleischmann's inchgower gordon's, "fleischmann's miltonduff." 
		\end{column}
	\end{columns}
\end{frame}

\subsection{Blocos}

\begin{frame}{Blocos}
	\begin{block}{Bloco}
		Tobermory bull shot wild turkey early times strega lynchburg lemonade haig \& haig "pinch" ruby dutchess. 
	\end{block}

	\begin{definition}
		Kentucky tavern manhattan sazerac glenfiddich kalimotxo smoky martini grant's.
	\end{definition}

\end{frame}

\begin{frame}{Mais blocos}
	\begin{alertblock}{Bloco de alerta}
		Kalimotxo critty bo glen moray mint julep jungle juice old crow mimosa early times tom and jerry. 
	\end{alertblock}
		
	\begin{exampleblock}{Bloco de exemplo}
		Ron rico batida speyburn, harper's colombia canadian club.
	\end{exampleblock}
\end{frame}

\begin{frame}{Blocos padrões da matemática}
	\begin{theorem}
		$ a^2 = b^2 + c^2 $
	\end{theorem}
	
	\begin{corollary}
		$\Delta = b^2 - 4ac$
	\end{corollary}
	
	\begin{proof}
		$\dfrac{\pi}{4} = 1 - \dfrac{1}{3} + \dfrac{1}{5} - \dfrac{1}{7} -\ldots$
	\end{proof}
\end{frame}

\subsection{Figuras}

\begin{frame}{Figuras}
	\begin{columns}
		\begin{column}{0.5\linewidth}
			\includegraphics[width=\linewidth]{logo_ufscar}
		\end{column}
		\begin{column}{0.5\linewidth}
			\includegraphics[width=\linewidth]{logo_nilc}
		\end{column}
	\end{columns}
\end{frame}

\section{Conclusões}
\begin{frame}
	\centering
	\textit{That's all folks!}
	
	\pause
	\alert{Questões?}
\end{frame}


\end{document}
