%! Author = Jander Moreira
%! Date = 08/05/2023
\documentclass[a4paper, 11pt]{article}
\usepackage[T1]{fontenc}

\usepackage{amsfonts}
\usepackage[brazilian, language = english]{algxpar__}
\usepackage[showframe]{geometry}
\geometry{top = 2.5cm, bottom = 2cm, right = 2.5cm, left = 4cm}
\usepackage[backref=page]{hyperref}
\hypersetup{
    colorlinks,
    urlcolor = blue!20!black,
    linkcolor = blue!10!black,
    citecolor = black!80,
}
\usepackage{imakeidx}
\makeindex
\usepackage[outputdir=../out]{minted}
% \usepackage{tcolorbox}
\tcbuselibrary{listings, minted}
\tcbset{
    colback = blue!3,
    sharp corners,
    fontlower = \footnotesize,
    minted options={
        fontsize = \footnotesize,
        breaklines,
        autogobble,
        % linenos,
        % numbersep = 3mm,
    },
    listing engine = minted,
    sidebyside,
}
\usepackage{tikz}
\usepackage{etoolbox}
\usepackage{textcomp}

\colorlet{argumentcolor}{brown!50!black}
\NewDocumentCommand{\Argument}{ m }{$\langle$\textcolor{argumentcolor}{\textsl{#1}}$\rangle$}
\NewDocumentCommand{\MandatoryArgument}{ m }{\texttt{\{}\Argument{#1}\texttt{\}}}
\NewDocumentCommand{\OptionalArgument}{ m }{\texttt{[}\Argument{#1}\texttt{]}}

\usepackage{color}

\newenvironment{commands}{
    \VerbatimEnvironment
    \tcolorbox[
    bottomrule = 0pt,
    toprule = 0pt,
    leftrule = 0pt,
    rightrule = 0pt,
    titlerule = 0pt,
    colframe = white,
    colback = white,
    sharp corners,
    pad at break*=1pc,
    enhanced jigsaw,
    overlay first and middle={
        \coordinate (A1) at ($(interior.south east) + (-10pt,5pt)$);
        \coordinate (C1) at ($(interior.south east) + (-6pt,7.5pt)$);
        \draw[fill=tcbcolframe] (A1) -- +(0,5pt) -- (C1) -- cycle;
    },
    ]%
    \Verbatim[gobble = 4, tabsize=4, commandchars = &\[\]]%
    }
%     {%
%     \endVerbatim%
%     \endtcolorbox%
% }

        \newtcolorbox{codigo}[1]{%
    bottomrule = 0pt,
    toprule = 0pt,
    leftrule = 0pt,
    rightrule = 0pt,
    titlerule = 0pt,
    sharp corners,
    sidebyside,
    sidebyside align = top,
    #1
}

\usetikzlibrary{calc}
\newenvironment{example}{
    \begingroup\tcbverbatimwrite{\jobname_code.tmp}
    }{
    \endtcbverbatimwrite\endgroup%
    \begin{tcolorbox}
        [
        bottomrule = 0pt,
        toprule = 0pt,
        leftrule = 0pt,
        rightrule = 0pt,
        titlerule = 0pt,
        colframe = white,
        segmentation style = {black!30, solid},
        colback = white,
        sharp corners,
        pad at break* = 1pc,
        enhanced jigsaw,
        breakable,
        overlay first and middle={
            \coordinate (A1) at ($(interior.south east) + (-10pt,5pt)$);
            \coordinate (C1) at ($(interior.south east) + (-6pt,7.5pt)$);
            \draw[fill=tcbcolframe] (A1) -- +(0,5pt) -- (C1) -- cycle;
        },
        ]%
        \small\VerbatimInput[tabsize=4, gobble = 2]{\jobname_code.tmp}
        \tcblower
% 	\begin{minipage}{0.85\linewidth}
        \begin{algorithmic}
            \input{\jobname_code.tmp}
        \end{algorithmic}
% 	\end{minipage}
    \end{tcolorbox}
}

% \NewDocumentCommand{\DefineMacro}{ m m }{
%     \begin{tikzpicture}%[overlay]
%         \node[fill = black!10, rounded corners = 2pt, font = \tiny] {#1};
%     \end{tikzpicture}
% }


% ShowMacro: showw macro name and description
\newmintinline{latex}{}
\NewDocumentCommand{\Macro}{ m }{\expandafter\latexinline\expandafter{\csname#1\endcsname}}
\NewDocumentEnvironment{macro}{ m O{} }{
    \index{#1@\texttt{\textbackslash#1}}
    \noindent
    $\triangleright$ ~\Macro{#1}#2
    \par
    }{
    \par
    \medskip
}

\NewDocumentCommand{\PackageName}{ m }{\mbox{\textsf{#1}}}


\title{The \PackageName{algxpar} package\thanks{This document corresponds to \PackageName{algxpar}~v\AlgVersion, dated \AlgDate.}}
\author{Jander Moreira\\\texttt{moreira.jander@gmail.com}}
\date{\today}

\begin{document}
\maketitle
\sloppy

% \begin{abstract}
%     The \PackageName{algxpar} packages is an extension of the \PackageName{algorithmicx} package to handle multiline text with the proper indentation.
% \end{abstract}
%
% \tableofcontents
% \vspace{2em}
%
% % \changes{v0.9}{2019/10/24}{First version}
% % \changes{v0.91}{2020/04/30}{Macro now can be used as super-/subscripts in math formulas, while still preventing hyphenaton in text mode.}
% % \changes{v0.91}{2020/06/14}{New macro for assignments, using $\gets$}
% % \changes{v0.91}{2020/06/14}{New macro for assignments (verbose)}
%
%
% \section{Introduction}
% I teach algorithms and programming and have adopted the \PackageName{algorithmicx} package (\PackageName{algpseudocode}) to write my algorithms as it provides clear and easy-to-read pseudocodes with a minimum of effort to achieve visually pleasing code.
%
% As part of the teaching process, I use very detailed commands in my algorithms before students start using a more synthetic form. For example, I initially write ``Start a counter $c$ with the value $0$'', which later becomes ``${c \gets 0}$''. This leads to sentences that often span text over two or more lines, especially in two-column documents with nested commands.
%
% Unfortunately, \PackageName{algorithmx} doesn't natively support multiline statements. This package therefore extends the macros to handle multiple lines properly. Some new commands and features were also added.
%
%
% \section{Package usage and options}
% To use the package simply add it to preamble.
%
% \usemintedstyle{tango}
% \newminted{latex}{autogobble}
%
% \begin{latexcode}
%     \usepackage{algxpar}
% \end{latexcode}
% % Package options:
% % \begin{itemize}
% %     \item[\texttt{noend}] Suppress the line at the end of a block.
% %     \item[\texttt{english}] Selects english keywords (default).
% %     \item[\texttt{brazilian}] Selects portuguese/brazilian keywords.
% % \end{itemize}
% %
% %
% % \section{Writting pseudocode}
% % All pseudocode is written inside the \texttt{algorithmic} environment.
% %
% % \subsection{Documentation}
% % A series of macros are defined to provide the header documentation for a pseudocode. The basics are:
% \begin{macro}{Description}
%     General description of the pseudocode.
% \end{macro}
% \begin{macro}{Require}
%     The required initial state that the code relies on. These are \textit{preconditions}.
% \end{macro}
% \begin{macro}{Ensure}
%     The final state produced by the code. These are \textit{preconditions}.
% \end{macro}
%
% \begin{tcblisting}{}
%     \begin{algorithmic}
%         \Description Calculation of the factorial of a natural number through successive multiplications
%         \Require $n \in \mathbb{N}$
%         \Ensure $f = n!$
%     \end{algorithmic}
% \end{tcblisting}
% %
% % Also are provided:
% % \DescribeMacro{Input}{Inputs.}
% % \DescribeMacro{Output}{Outputs}
% %
% % \begin{tcblisting}{}
% %     \begin{algorithmic}
% %         \Description Calculation of the factorial of a natural number through successive multiplications
% %         \Input $n$ (integer)
% %         \Output $n!$ (integer)
% %     \end{algorithmic}
% % \end{tcblisting}
% %
% % \subsection{Constants and Identifiers}
% %
% % Constants: \True, \False\ and \Nil.
% %
% % Identifiers: |\Id{}|.
% %
% % \subsection{Assignments and I/O}
% %
% % \subsection{Comments}
% %
% % \subsection{Statements}
% %
% % The macro |\State| can still be used and was left unchanged. A replacement statement macro is |\Statep|.
% %
% % \subsection{Flow Control}
% %
% % \subsection{Procedures and Functions}
% %
% % \begin{tcblisting}{}
% %     \begin{algorithmic}
% %         \While{\False}
% %         \EndWhile
% %         \If{\True ou \False}
% %             \Statep{Say ok}
% %         \ElsIf{$a>\alpha$}
% %         \ElsIf{$b>\beta$}[languages/else = intonces\ldots]
% %         \Else[language = brazilian]
% %         \EndIf[language = english]
% %         \ForAll{}
% %         \EndFor
% %     \end{algorithmic}
% % \end{tcblisting}
% %
% %
% % \section{Customization and Fine Tunning}
% %
% % %
% %
% \begin{macro}{Keyword}[\OptionalArgument{language}\MandatoryArgument{keyword}]
%     Macros like \Macro{algorithmicwhile} from \PackageName{algorithimicx} where replaced with \Macro{Keyword}, using \latexinline{\Keyword{while}} or \latexinline{\Keyword[brazilian]{while}}, for example.
%     If \Argument{language} is not specified, the current language is used. \Argument{keyword} is any keyword defined for a language, including custom ones.
%
%     Section~\ref{sec:languages} shows the predefined keywords.
%
%     \begin{tcblisting}{}
%         In English, the ``for all'' loop uses \Keyword{forall}, while it is \Keyword[brazilian]{forall} in Brazilian Portuguese.
%     \end{tcblisting}
% \end{macro}
%
% \begin{macro}{SetKeyword}[\OptionalArgument{language}\MandatoryArgument{keyword}\MandatoryArgument{value}]
%     The macro \Macro{SetKeyword} changes a given \Argument{keyword} to \Argument{value} if it exists; otherwise new keyword is created.
%     If \Argument{language} is not specified, then the current language is affected. Changes are local if the macro is used inside a group.
% \end{macro}
%
%
% %
% % \makeatletter
% % \RenewDocumentCommand{\If}{ O{} m O{} }{%
% %     \axp@ProcessCommandOptions{#1}{#3}%
% %     \ALG@makeentity{If}[\axp@CurrentComment]{#2}%
% % }
% % \appto{\EndIf}{\endgroup}
% %
% % \RenewDocumentCommand{\Switch}{ O{} m O{} }{%
% %     \axp@ProcessCommandOptions{#1}{#3}%
% %     \ALG@makeentity{Switch}[\axp@CurrentComment]{#2}%
% % }
% % \appto{\EndSwitch}{\endgroup}
% % \makeatother
% %
% %
% % \begin{algorithmic}
% %     \If{abc and \Nil}[language = brazilian, languages/end = fim do meu querido, comment is here]
% %         \Statep{teste}
% %     \EndIf%\endgroup
% %     \While{abc}
% %         \Statep{x}
% %     \EndWhile
% % \end{algorithmic}
% % %
% % % \AlgSet{
% % %     brazilian
% % % }
% % % \AlgLanguageSet{brazilian}{switch = selecione a partir de}
% % %
% % % \begin{algorithmic}
% % %     \Switch{asdf}[something in here a little longer, languages/switch = selecione]
% % %     \Statep{$v \gets \True$}
% % %     \EndSwitch
% % % \end{algorithmic}
% %

\begin{algorithmic}
    \While[asdf]{something}[adfsadfadf]
        \State x
    \EndWhile
\end{algorithmic}

\clearpage
\printindex

\end{document}